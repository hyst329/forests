\documentclass[12pt]{article}
\usepackage{unicode-math}
\setmainfont
[    Extension = .otf,
   UprightFont = *-regular,
      BoldFont = *-bold,
    ItalicFont = *-italic,
BoldItalicFont = *-bolditalic,
]{xits}

\setmathfont
[    Extension = .otf,
      BoldFont = *bold,
]{xits-math}
\usepackage[scaled=.95]{sourcecodepro}
\usepackage{polyglossia}
\usepackage{verbatim}
\usepackage{graphicx}
\usepackage{hyperref}
\usepackage[margin=13mm]{geometry}
\setmainlanguage{russian}
\begin{document}
\title{Отчет по дисциплине <<Статистические пакеты>>}
\author{Выполнил: студент группы 09--616 Галанин Дмитрий \and Преподаватель: Заикин Артем Александрович}
\date{18. 12. 2016}
\maketitle
\section{Постановка задачи}
Исходные данные представляют собой набор из 15120 наблюдений следующего вида: каждое наблюдение - участок леса размером $30 \times 30$ метров из Национального Леса имени Рузвельта.

Описание полей:
\begin{itemize}
	\item Elevation --- средняя высота над уровнем моря,
	\item Aspect --- азимут направления самого крутого спуска на участке,
	\item Slope --- подъем в градусах,
	\item Horizontal\_Distance\_To\_Hydrology --- горизонтальное расстояние до ближайшего источника воды,
	\item Vertical\_Distance\_To\_Hydrology --- вертикальное расстояние до ближайшего источника воды,
	\item Horizontal\_Distance\_To\_Roadways --- горизонтальное расстояние до ближайшей дороги,
	\item Hillshade\_9am (0 to 255 index) --- количество тени в 9:00,
	\item Hillshade\_Noon (0 to 255 index) --- количество тени в 12:00,
	\item Hillshade\_3pm (0 to 255 index) --- количество тени в 15:00,
	\item Horizontal\_Distance\_To\_Fire\_Points --- горизонтальное расстояние до ближайшего участка распространения огня,
	\item Wilderness\_Area (4 бинарных показателя) --- категория дикости местности,
	\item Soil\_Type (40 бинарных показателей) --- тип почвы,
	\item Cover\_Type* (7 типов, указаны целым числом от 1 до 7) --- тип лесного покрова.
\end{itemize}
Задачей является построение регрессии переменной Cover\_Type.
\section{Анализ и обработка данных}
Вначале, после загрузки набора данных (<<датафрейма>>, англ. \textit{dataframe})
выполним преобразование бинарных показателей (а также переменной Cover\_Type) в категориальные переменные.
(Создаются новые столбцы Soil\_Type и Wilderness\_Area, содержащие в числовом виде тип (категорию) переменной вместо
бинарных показателей, после чего <<старые>> столбцы удаляются. Это упростит нам запись формул при создании моделей.)

Анализ данных с помощью функций типа \verb|boxplot()| показывает, что <<выбросов>> (англ. \textit{outliers}) нет или
имеется достаточно небольшое количество (не более 5,5\%; результаты представлены в таблице),
что не должно сказаться на точности получаемых моделей.

\begin{tabular}{|l|r|}
	\hline
	Переменная                   & Число <<выбросов>> \\ \hline
	Elevation                              & 0                               \\
	Aspect                                 & 0                               \\
	Slope                                  & 57                              \\
	Horizontal\_Distance\_To\_Hydrology    & 512                             \\
	Vertical\_Distance\_To\_Hydrology      & 586                             \\
	Horizontal\_Distance\_To\_Roadways     & 830                             \\
	Hillshade\_9am                         & 408                             \\
	Hillshade\_Noon                        & 393                             \\
	Hillshade\_3pm                         & 124                             \\
	Horizontal\_Distance\_To\_Fire\_Points & 645                             \\ \hline
\end{tabular}

\section{Выбор моделей}
В нашем примере зависимая переменная является категориальной. Поэтому при построении моделей будем использовать логистическую
регрессию (параметр \verb|family=poisson(link="log")|).

Вначале необходимо <<отсеять>> переменные, от которых (с большой вероятностью) не зависит наша переменная Cover\_Type. Для этого
используем GLM с формулой вида \verb|Cover_Type ~ .|:

(Точка в формуле после тильды обозначает зависимость от всех остальных (не указанных явно) переменных).

Далее по значениям тестовой $z$-статистики (и вероятности подтверждения гипотезы $H_0$, утверждающей о том, что коэффициент при
данной переменной-регрессоре в модели равен 0, т. е. зависимость фактически отсутствует) выбираем существенные переменные в модели.


Следует заметить, что графики, как правило, предназначены для линейных регрессионных моделей и не имеют смысла для логистических
(см. также здесь:\\ http://stats.stackexchange.com/questions/121490/interpretation-of-plotglm-model).
Поэтому для валидации модели обратимся непосредственно к численным данным (функции \verb|summary()|)
\begin{verbatim}
Call:
glm(formula = Cover_Type ~ Elevation + Aspect + Hillshade_9am +
    Hillshade_Noon + Soil_Type + Horizontal_Distance_To_Fire_Points +
    Wilderness_Area, family = poisson(link = "log"), data = df)

Deviance Residuals:
    Min       1Q   Median       3Q      Max
-2.9961  -0.6153  -0.0370   0.4909   3.1695

Coefficients:
                                     Estimate Std. Error z value Pr(>|z|)
(Intercept)                         1.835e+00  9.677e-02  18.962  < 2e-16 ***
Elevation                          -3.259e-04  3.231e-05 -10.084  < 2e-16 ***
Aspect                              1.780e-04  5.091e-05   3.496 0.000473 ***
Hillshade_9am                       1.301e-03  1.937e-04   6.716 1.87e-11 ***
Hillshade_Noon                     -7.732e-04  2.193e-04  -3.526 0.000423 ***
Soil_Type2                          2.240e-02  3.368e-02   0.665 0.505951
Soil_Type3                         -9.175e-02  3.101e-02  -2.958 0.003093 **
Soil_Type4                         -9.774e-02  3.363e-02  -2.906 0.003656 **
Soil_Type5                          9.782e-02  4.514e-02   2.167 0.030253 *
Soil_Type6                          4.303e-02  3.314e-02   1.298 0.194169
Soil_Type8                         -4.170e-01  7.082e-01  -0.589 0.555988
Soil_Type9                         -5.008e-01  2.322e-01  -2.157 0.031013 *
Soil_Type10                         1.508e-01  2.943e-02   5.123 3.00e-07 ***
Soil_Type11                         2.243e-02  3.828e-02   0.586 0.557867
Soil_Type12                        -4.762e-01  6.047e-02  -7.875 3.42e-15 ***
Soil_Type13                         1.278e-01  3.867e-02   3.304 0.000954 ***
Soil_Type14                         6.239e-02  4.493e-02   1.388 0.165011
Soil_Type16                         7.981e-02  5.338e-02   1.495 0.134832
Soil_Type17                         6.346e-02  3.293e-02   1.927 0.053982 .
Soil_Type18                         1.957e-01  7.365e-02   2.657 0.007892 **
Soil_Type19                        -1.356e-01  9.385e-02  -1.445 0.148359
Soil_Type20                        -1.864e-01  5.933e-02  -3.141 0.001685 **
Soil_Type21                        -3.239e-01  1.624e-01  -1.995 0.046080 *
Soil_Type22                        -7.245e-01  5.878e-02 -12.326  < 2e-16 ***
Soil_Type23                        -1.994e-01  4.179e-02  -4.771 1.83e-06 ***
Soil_Type24                        -3.293e-01  5.330e-02  -6.179 6.45e-10 ***
Soil_Type25                        -4.837e-01  7.086e-01  -0.683 0.494826
Soil_Type26                        -1.308e-01  8.028e-02  -1.630 0.103195
Soil_Type27                        -2.788e-01  1.668e-01  -1.671 0.094640 .
Soil_Type28                        -2.110e-01  1.956e-01  -1.079 0.280648
Soil_Type29                        -1.074e-01  4.195e-02  -2.560 0.010464 *
Soil_Type30                         2.796e-01  4.150e-02   6.736 1.63e-11 ***
Soil_Type31                        -2.225e-01  4.674e-02  -4.760 1.93e-06 ***
Soil_Type32                        -2.542e-01  4.160e-02  -6.111 9.91e-10 ***
Soil_Type33                        -1.327e-01  4.071e-02  -3.259 0.001119 **
Soil_Type34                         3.175e-02  1.142e-01   0.278 0.780940
Soil_Type35                         7.891e-01  5.562e-02  14.187  < 2e-16 ***
Soil_Type36                         6.155e-01  1.356e-01   4.539 5.66e-06 ***
Soil_Type37                         9.807e-01  7.883e-02  12.440  < 2e-16 ***
Soil_Type38                         7.302e-01  4.319e-02  16.906  < 2e-16 ***
Soil_Type39                         6.998e-01  4.348e-02  16.097  < 2e-16 ***
Soil_Type40                         7.741e-01  4.733e-02  16.355  < 2e-16 ***
Horizontal_Distance_To_Fire_Points  1.808e-05  4.949e-06   3.653 0.000259 ***
Wilderness_Area2                    1.570e-01  2.830e-02   5.547 2.91e-08 ***
Wilderness_Area3                    2.923e-01  1.880e-02  15.544  < 2e-16 ***
Wilderness_Area4                    1.444e-01  2.603e-02   5.549 2.88e-08 ***
---
Signif. codes:  0 ‘***’ 0.001 ‘**’ 0.01 ‘*’ 0.05 ‘.’ 0.1 ‘ ’ 1

(Dispersion parameter for poisson family taken to be 1)

    Null deviance: 16546  on 15119  degrees of freedom
Residual deviance: 10543  on 15074  degrees of freedom
AIC: 57760

Number of Fisher Scoring iterations: 5
\end{verbatim}
Зедсь уже видно, что данная модель не имеет проблем --- все переменные значимы (p-values для них достаточно малы;
переменная Soil\_Type же являеся категориальной), значение
Residual Deviance адекватно (значение \verb|pchisq(10543, 15074)| пренебрежимо мало --- $7.1 \times 10^{-189}$).
Кроме того, с потерей всего 45 ($15119-15074$) степеней свободы отклонение существенно (на 36.3\%, с 16546 до 10543) уменьшилось.
Поэтому мы можем точно сказать. что принимаем данную модель.

Рассмотрим также модель GAM (для нее пришлось исключить еще часть предположительно незначимых переменных, а также ввести гладкость
по переменной Elevation):
\begin{verbatim}
Family: poisson
Link function: log

Formula:
Cover_Type ~ s(Elevation) + Aspect + Vertical_Distance_To_Hydrology +
    Hillshade_Noon + Hillshade_3pm + Wilderness_Area

Parametric coefficients:
                                 Estimate Std. Error z value Pr(>|z|)
(Intercept)                     1.686e+00  4.465e-02  37.759  < 2e-16 ***
Aspect                          1.295e-04  4.839e-05   2.676  0.00745 **
Vertical_Distance_To_Hydrology -4.712e-04  7.228e-05  -6.519 7.09e-11 ***
Hillshade_Noon                 -1.901e-03  2.318e-04  -8.201 2.39e-16 ***
Hillshade_3pm                  -3.033e-04  1.379e-04  -2.200  0.02782 *
Wilderness_Area2               -7.981e-02  2.571e-02  -3.105  0.00190 **
Wilderness_Area3                1.577e-01  1.194e-02  13.207  < 2e-16 ***
Wilderness_Area4                1.251e-01  2.706e-02   4.623 3.78e-06 ***
---
Signif. codes:  0 ‘***’ 0.001 ‘**’ 0.01 ‘*’ 0.05 ‘.’ 0.1 ‘ ’ 1

Approximate significance of smooth terms:
               edf Ref.df Chi.sq p-value
s(Elevation) 8.981      9   4957  <2e-16 ***
---
Signif. codes:  0 ‘***’ 0.001 ‘**’ 0.01 ‘*’ 0.05 ‘.’ 0.1 ‘ ’ 1

R-sq.(adj) =  0.391   Deviance explained = 38.9%
UBRE = -0.32875  Scale est. = 1         n = 15120
\end{verbatim}
AIC данной модели равен 57273.77.

Как видим, GAM-модель также годится для описания зависимой переменной Cover\_Type.
\section{Листинг кода на R}
\verbatiminput{../forests.r}
\end{document}
